\section*{Lab activity 2}

\subsection*{Learning objectives}
After finishing this lab activity you will be able to:
\begin{itemize}
  \item Implement a cluster of remote controllers inside Mininet using the Python
  middle-level Mininet API
  \item Test the network connectivity and the performance of a network which
  includes a cluster of remote controllers
  \item Understand the main functions provided by the middle-level Mininet API required
  to implement a cluster of remote controllers
\end{itemize}






\subsection*{Scenario}
In this activity you will implement the simple topology shown in figure \ref{fig:topology-1} using
a Python script and the middle-level API provided by Mininet. The two controllers
shown in the topology diagram will be local controllers for this activity.
The topology has two different switches: each one will be connected to a different
local controller.

Begin by creating a new Python script, then import Mininet classes required for
this activity and define the function that will be used to create the topology.
Inside the body of this function, create a new Mininet netowrk and add to it the
required hosts, switches, links and controllers. After writing the script, execute
it to create the network and test its connectivity and performance.

This lab activity assumes you are proficient in [...]. A basic knowledge of the
Python programming language is also assumed.
