\section{Task 1: build a cluster of local controllers}
In this task we will create a simple network topology with multiple local controllers
(that make up a cluster???).

The topology is the simple linear topology shown in figure *** and it will be
created with a python script using the middle-level API provided by mininet. The
topology includes two switches, each one connected to a different local controller.

The script shown in listing \ref{lst:task-1-template-script} will be used as reference
in the following steps in order to progressively write the complete python script
required for creating the proposed topology. The script simply creates a new empty mininet network and starts the mininet CLI:
we will use it as a start point and in the next steps and we will add code for creating hosts,
switches, links and controllers. In the first line of the body of the function \code{multiControllerNet}
a new mininet network is created invoking the Mininet constructor: the parameters
passed to the constructor are the Controller class and the OVSSwitch class, therefore
the used controllers will be the Stanford/OpenFlow reference controllers while
the used switches will be Open vSwitch switches. Note that these two classes are
the default parameters in the Mininet constructor, so it is not really necessary
to specify them.

\begin{minipage}{\linewidth}
\begin{lstlisting}[label=lst:task-1-template-script, caption=task 1 reference python script]
#!/usr/bin/python
from mininet.net import Mininet
from mininet.node import Controller, OVSSwitch
from mininet.cli import CLI
from mininet.log import setLogLevel, info

def multiControllerNet():
    net = Mininet( controller=Controller, switch=OVSSwitch )

    "" Create hosts ""



    "" Create switches ""



    "" Create controllers ""



    info( "*** Running CLI\n" )
    CLI( net )

    info( "*** Stopping network\n" )
    net.stop()

if __name__ == '__main__':
    setLogLevel( 'info' )  # for CLI output
    multiControllerNet()
\end{lstlisting}
\end{minipage}



\subsection*{Step 1: create a new python script}
First step is creating a new text file and copying into it the content of the
python script shown above. Save the file as \emph{controllers-1.py} in your
custom directory inside your mininet virtual machine.


\subsection*{Step 2: add host to the network}
The second step is to create the hosts of the network. To do that, we need to add
the following lines of code:

 \code{info( "*** Creating hosts\textbackslash n" )} \\
 \code{h1 = net.addHost('h3')} \\
 \code{h2 = net.addHost('h4')} \\
 \code{h3 = net.addHost('h5')} \\
 \code{h4 = net.addHost('h6')}

The first line prints to the console that hosts are being created while the others
lines add four new hosts to the network respectively called h3, h4, h5 and h6.



\subsection*{Step 3: add switches to the network}
Now we need to add the two switches to the network, therefore we have to add the
following code to the respective section of the template script:

\code{info( "*** Creating switches\textbackslash n" )} \\
\code{s1 = net.addSwitch( 's1' )} \\
\code{s2 = net.addSwitch( 's2' )}

As in the step 2, the first line print to the console a message to inform the user
that the switches are being created, while the next two lines add two new switches
to the network.



\subsection*{Step 4: create links between nodes}
The next step is adding the links between hosts and switches and the link between the two
switches. For doing that we add the following lines:

\code{info( "*** Creating links\textbackslash n" )}
\code{net.addLink( h3, s1 )} \\
\code{net.addLink( h4, s1 )} \\
\code{net.addLink( h5, s2 )} \\
\code{net.addLink( h6, s2 )} \\
\code{net.addLink( s1, s2 )}



\subsection*{Step 5: create controllers} \label{sec:step-5}
In this step we are going to create two local SDN controllers. The code we have to
add to the script is the following:

\code{info( "*** Creating (reference) controllers\textbackslash n" )} \\
\code{c1 = net.addController( 'c1', port=6633 )} \\
\code{c2 = net.addController( 'c2', port=6634 )}

Note that we specified a different TCP port for each controller (the controllers
will listen on the specified port for switches that want to set up a connection).

\textbf{Why did we specify a different port for each controller?}

\hrulefill

\hrulefill
% Because each swtich has to setup one TCP connection to each controller and
% it's not possible having more than one TCP connection on the same port, so
% specifying two different ports makes it possible to connect one single switch
% to multiple controllers.


\subsection*{Step 6: start the network}
After adding all the required nodes to the network we can finally start it. In order
to do that, we have to build the mininet network and start the controllers and the
switches:

\code{info( "*** Starting network\textbackslash n" )} \\
\code{net.build()} \\
\code{c1.start()} \\
\code{c2.start()} \\
\code{s1.start( [ c1 ] )} \\
\code{s2.start( [ c2 ] )}

In the last two lines of code we used the function start() to start the two switches,
passing as parameter

After this step, the python script required to build the topology for the task one
is completed. The full script is shown in listing \ref{lst:task-1-complete-script}.



\subsection{Step 7: test network connectivity and performance}
The final step is execute the script and test the created topology: try to verify
the network connectivity between all hosts and the bandwidth between \code{h3} and \code{h6}.
Write in the lines below the commands you used and the results you obtained.

\hrulefill

\hrulefill

\hrulefill

\begin{lstlisting}[label=lst:task-1-complete-script, caption=Task 1 complete python script]
#!/usr/bin/python
from mininet.net import Mininet
from mininet.node import Controller, OVSSwitch
from mininet.cli import CLI
from mininet.log import setLogLevel, info

def multiControllerNet():
    net = Mininet( controller=Controller, switch=OVSSwitch )

    info( "*** Creating hosts\n" )
    h1 = net.addHost('h3')
    h2 = net.addHost('h4')
    h3 = net.addHost('h5')
    h4 = net.addHost('h6')

    info( "*** Creating switches\n" )
    s1 = net.addSwitch( 's1' )
    s2 = net.addSwitch( 's2' )

    info( "*** Creating links\n" )
    net.addLink( h3, s1 )
    net.addLink( h4, s1 )
    net.addLink( h5, s2 )
    net.addLink( h6, s2 )
    net.addLink( s1, s2 )

    info( "*** Creating (reference) controllers\n" )
    c1 = net.addController( 'c1', port=6633 )
    c2 = net.addController( 'c2', port=6634 )

    info( "*** Starting network\n" )
    net.build()
    c1.start()
    c2.start()
    s1.start( [ c1 ] )
    s2.start( [ c2 ] )

    info( "*** Running CLI\n" )
    CLI( net )

    info( "*** Stopping network\n" )
    net.stop()

if __name__ == '__main__':
    setLogLevel( 'info' )  # for CLI output
    multiControllerNet()
\end{lstlisting}
