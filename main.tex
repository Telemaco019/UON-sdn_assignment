\documentclass[12pt, a4paper]{article}
\usepackage[utf8]{inputenc}
\usepackage{listings}
\usepackage{courier}
\usepackage{fancyhdr}
\usepackage{color}
\usepackage{xcolor}
\usepackage{caption}
\usepackage{graphicx}
\usepackage{url}
\usepackage[titletoc]{appendix}
\usepackage{textcomp}
\usepackage{pdfpages}
\usepackage[colorlinks]{hyperref}
\hypersetup{urlcolor=blue}



\usepackage[british,english]{babel}
\usepackage[style=authoryear,dashed=false]{biblatex}
\addbibresource{bibliography.bib}
\renewcommand*{\nameyeardelim}{\addcomma\space}
\renewcommand*{\bibfont}{\footnotesize}
\setlength\bibitemsep{2\itemsep}
\DeclareFieldFormat{urldate}{%
  (Accessed: \thefield{urlday}\addspace%
  \mkbibmonth{\thefield{urlmonth}}\addspace%
  \thefield{urlyear}\isdot)
}
\DeclareFieldFormat{url}{Available at: \url{#1}}



\definecolor{light-gray}{gray}{0.95}

\lstset{
 upquote=true,
 showspaces=false,
 showtabs=false,
 frame=none,
 tabsize=2,
 breaklines=true,
 numbers=none,
 showstringspaces=false,
 breakatwhitespace=true,
 escapeinside={(*@}{@*)},
 keywordstyle=\bfseries,
 basicstyle=\footnotesize\ttfamily,
}

\newcommand{\code}[1]{{\footnotesize\texttt{#1}}}

\setlength\parindent{0pt} % Indentazione paragrafi
\setlength{\parskip}{1ex plus 0.5ex minus 0.2ex} % Spaziatura paragrafi


\title{
  \includegraphics[width=0.8\textwidth]{img/logo.png}~ \\
  SDN controller clustering \\ \large Computer Networks module - SDN assignment
}
\author{Michele Zanotti}
\date{Spring term 2018}

\pagestyle{fancy}
\fancyhf{}
\lhead{Computer Networks}
\rhead{SDN assignment}
\rfoot{\thepage}

\begin{document}

\maketitle
\begin{figure}[htb]
	\centering
	\includegraphics[width=1\linewidth]{img/valuation-table.png}
\end{figure}

\newpage





\section*{Overview}
In this paper are proposed four different lab activities meant to teach how to implement
a cluster of controllers inside Mininet using some of the most common available methods.
Each activity focuses on a single method and uses it to implement a given topology.

In particular, in each activity the following methods will be explained:
\begin{itemize}
  \item \textbf{Activity 1}: implement a network with multiple controllers using a python
  script and the middle-level Mininet API
  \item \textbf{Activity 2}: implement a network with multiple local controllers using a python
  script, a custom switch class and the topology classes provided by the Mininet high-level API
  \item \textbf{Activity 3}: dynamically connect the switches of a Mininet
  network to different controllers using the tool ovs-vsctl.
\end{itemize}

The paper also includes a fourth lab activity (\textbf{Activity 4}), which is a conclusive challenge meant
to let the reader test the knowledge acquired with the execution of the previous
four activities. Each activity moreover contains some final questions aimed to make the reader
reflect on what he did during that specific activity and possibly make him try to
bring minor changes to the implemented topology. The solution for the activity 4
and the answers to the questions included in the previous activities can be
found in Appendix A.

It's important to point out that all the proposed activities are meant to be
only an introduction to SDN controller clustering, therefore this paper won't cover
some of the aspects that have to be considered when implementing a cluster of
controllers in a real-world context, such as load balancing and communication
between controllers. More information about these topics can be found in references
\cite{ref-2}, \cite{ref-1} and \cite{ref-9}.
The communication between different controllers can however be automatically managed
by Mininet, the network emulator used in these paper activities. When in fact a
network with multiple local controllers is implemented inside the emulator, all
the local controllers are executed inside the kernel space on a single machine
and therefore they can communicate with each other without an external
communication mechanism being necessary.

\section*{Lab activity 1}





\subsection*{Learning objectives}
After finishing this lab activity you will be able to:
\begin{itemize}
  \item Implement a cluster of local controllers inside Mininet using the Python
  middle-level Mininet API
  \item Test the network connectivity and the performance of a network which
  includes a cluster of local controllers
  \item Understand the main functions provided by the middle-level Mininet API required
  to implement a cluster of local controllers
  \item Reflect about the reasons of using more then one controller in SDN networks
\end{itemize}






\subsection*{Scenario}
In this activity you will implement the simple topology shown in figure \ref{fig:topology-1} using
a Python script and the middle-level API provided by Mininet. The two controllers
shown in the topology diagram will be local controllers for this activity.
The topology has two different switches: each one will be connected to a different
local controller.

Begin by creating a new Python script, then import Mininet classes required for
this activity and define the function that will be used to create the topology.
Inside the body of this function, create a new Mininet netowrk and add to it the
required hosts, switches, links and controllers. After writing the script, execute
it to create the network and test its connectivity and performance.

This lab activity assumes you are proficient in [...]. A basic knowledge of the
Python programming language is also assumed.





\subsection*{Task 1: write the skeleton of the Python script}
\subsubsection*{Step 1}
Create a new Python script and edit with the text editor you prefer. If your editing
it inside the Mininet virtual machine, it is suggested to use Vim text editor.

\subsubsection*{Step 2}
Import the Python classes from the Mininet API:
\begin{lstlisting}
#!/usr/bin/Python
from mininet.net import Mininet
from mininet.node import Controller, OVSSwitch
from mininet.cli import CLI
from mininet.log import setLogLevel, info
\end{lstlisting}

\subsubsection*{Step 3}
Make the script executable onyl as a program, set the verbosity level to ``info''
and call the function \code{multiControllerNet()}, which will be defined in the next
step and will be used to create the required topology:
\begin{lstlisting}
if __name__ == '__main__':
    setLogLevel( 'info' )
    multiControllerNet()
\end{lstlisting}

\subsubsection*{Step 4}
Define the function that will be used to create the topology:
\begin{lstlisting}
def multiControllerNet():
\end{lstlisting}

\subsubsection*{Step 6}
Inside the body of the function \code{multiControllerNet()} create a new Mininet
network:
\begin{lstlisting}
net = Mininet( controller=Controller, switch=OVSSwitch )
\end{lstlisting}

The Mininet network is created invoking the Mininet constructor: the parameters
passed to the constructor are the Controller class and the OVSSwitch class, therefore
the Stanford/OpenFlow reference controllers and Open vSwitch switches will be used
in the network we are goind to create. Note that these two classes are
the default parameters in the Mininet constructor, so it is not really necessary
to specify them.

\subsubsection*{Step 7}
Save the text file as ``\emph{controllers-1.py}'' in your custom directory inside
the mininet virtual machine.





\subsection*{Task 2: add hosts to the network}
\subsubsection*{Step 1}
Inside the body of the function \code{multiControllerNet()} add the following line
of code in order to print to the console that hosts are being created:
\begin{lstlisting}
info( "*** Creating hosts \n" )
\end{lstlisting}

\subsubsection*{Step 2}
Still inside the body of the function \code{multiControllerNet()}, create the
four hosts required for the topology by adding them to the mininet network
previously created:
\begin{lstlisting}
h1 = net.addHost('h3')
h2 = net.addHost('h4')
h3 = net.addHost('h5')
h4 = net.addHost('h6')
\end{lstlisting}
The function used to add the hosts to the network is \code{addHost('name')}, which
accept as parameter the name of the host that will be created. The hosts names in
this network therefore will be \code{h3}, \code{h4}, \code{h5} and \code{h6}.





\subsection*{Task 3: add switches to the network}
\subsubsection*{Step 1}
Inside the body of the function \code{multiControllerNet()} add the following line
of code in order to print to the console that switches are being created:
\begin{lstlisting}
info( "*** Creating switches \n" )
\end{lstlisting}

\subsubsection*{Step 2}
Still inside the body of the function \code{multiControllerNet()}, create the
two switches required for the topology by adding them to the mininet network
previously created:
\begin{lstlisting}
s1 = net.addSwitch('s1')
s2 = net.addSwitch('s2')
\end{lstlisting}







\subsection*{Task 4: create links between nodes}
\subsubsection*{Step 1}
Inside the body of the function \code{multiControllerNet()} add the following line
of code in order to print to the console that links are being created:
\begin{lstlisting}
info( "*** Creating links \n" )
\end{lstlisting}

\subsubsection*{Step 2}
Still inside the body of the function \code{multiControllerNet()}, create the
links between the hosts and the switches and the link between the two switches:
\begin{lstlisting}
net.addLink( h3, s1 )
net.addLink( h4, s1 )
net.addLink( h5, s2 )
net.addLink( h6, s2 )
net.addLink( s1, s2 )
\end{lstlisting}






\subsection*{Task 5: create the controllers}
\subsubsection*{Step 1}
Inside the body of the function \code{multiControllerNet()} add the following line
of code in order to print to the console that reference controllers are being created:
\begin{lstlisting}
info( "*** Creating (reference) controllers \n" )
\end{lstlisting}

\subsubsection*{Step 2}
Still inside the body of the function \code{multiControllerNet()}, add to the
Mininet network the two required controller:
\begin{lstlisting}
c0 = net.addController( 'c0', port=6633 )
c1 = net.addController( 'c1', port=6634 )
\end{lstlisting}

The function use to create the controllers is \code{addController}, which accept
as parameters the name of the controller which will be created and the TCP port that
will be used by the switches for connecting to the controller.







\subsection*{Task 6: start the mininet network}
\subsubsection*{Step 1}
Append to the body the function \code{multiControllerNet()} the following line
of code for printing to the console that the network is being started:
\begin{lstlisting}
info( "*** Starting network \n" )
\end{lstlisting}

\subsubsection*{Step 2}
Build the Mininet network:
\begin{lstlisting}
net.build()
\end{lstlisting}

\subsubsection*{Step 3}
Start the controllers:
\begin{lstlisting}
c0.start()
c1.start()
\end{lstlisting}

\subsubsection*{Step 4}
Start the switches, specifying for each switch the controller to wich connect:
\begin{lstlisting}
s1.start( [ c0 ] )
s2.start( [ c1 ] )
\end{lstlisting}

\subsubsection*{Step 5}
Start the Mininet CLI:
\begin{lstlisting}
info( "*** Running CLI\n" )
CLI( net )
\end{lstlisting}

\subsubsection*{Step 6}
Stop the network so that after the user exits the Mininet CLI the network is
stopped:
\begin{lstlisting}
info( "*** Stopping network\n" )
net.stop()
\end{lstlisting}

% Because each swtich has to setup one TCP connection to each controller and
% it's not possible having more than one TCP connection on the same port, so
% specifying two different ports makes it possible to connect one single switch
% to multiple controllers.



\subsection*{Task 7: execute the script and test the network}
After finishing the task 6 the script for implementing the required topology is
completed. The full script is shown in listing \ref{lst:activity-1-script} at the
bottom of this activity.

\subsubsection*{Step 1}
Execute the script as root: \\
\code{\$ sudo python controllers1.py}

\subsubsection*{Step 2}
Test the created topology: verify the network connectivity between all hosts
and the bandwidth between \code{h3} and \code{h6}. Write in the lines below the
commands you used and the results you obtained.

\hrulefill

\hrulefill

\hrulefill

\hrulefill





\subsection*{Task 8: reflection}
\subsubsection*{1 - What are the advantages of having more controllers instead of one
single controller which serves all the switches of the network?}
\hrulefill

\hrulefill

\hrulefill

\hrulefill

\subsubsection*{2 - Would the fault tolerance of the netowrk shown in figure
\ref{fig:topology-1} change if we used only one controller instead of two?}
\hrulefill

\hrulefill

\hrulefill

\hrulefill

\subsubsection*{3 - How could we improve the fault tolerance of the netowrk shown
in figure \ref{fig:topology-1}?} % Each switch connected to both controllers, each controller
% could be a distributed controller
\hrulefill

\hrulefill

\hrulefill

\hrulefill

\lstset{
 showspaces=false,
 showtabs=false,
 frame=single,
 tabsize=2,
 breaklines=true,
 numbers=left,
 showstringspaces=false,
 breakatwhitespace=true,
 escapeinside={(*@}{@*)},
 keywordstyle=\bfseries,
 basicstyle=\scriptsize\ttfamily,
}

\begin{minipage}{\linewidth}
\begin{lstlisting}[label=lst:activity-1-script, caption=complete Python script required for Activity 1]
#!/usr/bin/Python
from mininet.net import Mininet
from mininet.node import Controller, OVSSwitch
from mininet.cli import CLI
from mininet.log import setLogLevel, info

if __name__ == '__main__':
    setLogLevel( 'info' )
    multiControllerNet()

def multiControllerNet():
    net = Mininet( controller=Controller, switch=OVSSwitch )

    info( "*** Creating hosts\n" )
    h1 = net.addHost('h3')
    h2 = net.addHost('h4')
    h3 = net.addHost('h5')
    h4 = net.addHost('h6')

    info( "*** Creating switches\n" )
    s1 = net.addSwitch( 's1' )
    s2 = net.addSwitch( 's2' )

    info( "*** Creating links\n" )
    net.addLink( h3, s1 )
    net.addLink( h4, s1 )
    net.addLink( h5, s2 )
    net.addLink( h6, s2 )
    net.addLink( s1, s2 )

    info( "*** Creating (reference) controllers\n" )
    c0 = net.addController( 'c0', port=6633 )
    c1 = net.addController( 'c1', port=6634 )

    info( "*** Starting network\n" )
    net.build()
    c0.start()
    c1.start()
    s1.start( [ c0 ] )
    s2.start( [ c1 ] )

    info( "*** Running CLI\n" )
    CLI( net )

    info( "*** Stopping network\n" )
    net.stop()
\end{lstlisting}
\end{minipage}

\section*{Lab activity 2}

\subsection*{Learning objectives}
After finishing this lab activity you will be able to:
\begin{itemize}
  \item Implement a cluster of remote controllers inside Mininet using the Python
  middle-level Mininet API
  \item Test the network connectivity and the performance of a network which
  includes a cluster of remote controllers
  \item Understand the main functions provided by the middle-level Mininet API required
  to implement a cluster of remote controllers
\end{itemize}






\subsection*{Scenario}
In this activity you will implement the simple topology shown in figure \ref{fig:topology-1} using
a Python script and the middle-level API provided by Mininet. The two controllers
shown in the topology diagram will be local controllers for this activity.
The topology has two different switches: each one will be connected to a different
local controller.

Begin by creating a new Python script, then import Mininet classes required for
this activity and define the function that will be used to create the topology.
Inside the body of this function, create a new Mininet netowrk and add to it the
required hosts, switches, links and controllers. After writing the script, execute
it to create the network and test its connectivity and performance.

This lab activity assumes you are proficient in [...]. A basic knowledge of the
Python programming language is also assumed.

\lstset{
 upquote=true,
 showspaces=false,
 showtabs=false,
 frame=none,
 tabsize=2,
 breaklines=true,
 numbers=none,
 showstringspaces=false,
 breakatwhitespace=true,
 escapeinside={(*@}{@*)},
 keywordstyle=\bfseries,
 basicstyle=\footnotesize\ttfamily,
}






\section*{Lab activity 3}
\subsection*{Topology diagram}
\begin{figure}[htb]
	\centering
	\includegraphics[width=1\linewidth]{img/topology-3.png}
	\caption{the simple linear topology that will be implemented in this activity. It is
  assumed that c1 and c2 are remote controllers running on the same machine
  on which Mininet is running, respectively on the TCP ports 6634 and 6635,
  while c0 is a local controller.}
	\label{fig:topology-3}
\end{figure}







\subsection*{Learning objectives}
After finishing this lab activity you will be able to:
\begin{itemize}
  \item Use the tool ovs-vsctl for implementing a cluster of controllers inside Mininet
  \item Test the network connectivity and the performance of a network with multiple
  controllers
  \item Dynamically set the controller for any switch of a generic running Mininet
  network using the tool ovs-vsctl.
\end{itemize}






\subsection*{Scenario}
In this activity you will implement the topology shown in figure \ref{fig:topology-3}
using the command \code{mn} to create and start a new Mininet network, passing
\code{--topo} as parameter to specify the required topology. You will then
use the tool ovs-vsctl in order to set the controller for each switch according
to the topology diagram shown in figure \ref{fig:topology-3}.

Begin by creating and starting a new Mininet network using the command \code{mn}.
Once the network is running, start the remote controllers and use the tool ovs-vsctl
for setting the controller for each switch according to the topology diagram.
To conclude, test the network verifying the connectivity between all hosts.

This lab activity assumes that:
\begin{itemize}
  \item you are proficient in SDN networks
  \item you are proficient in Mininet network emulator
  \item you have already completed the previous lab activities of this paper.
\end{itemize}






\subsection*{Task 1: create and start the network}
Create and start a new Mininet network, using the parameter \code{--topo} to
specify the required topology:
\begin{lstlisting}
$ sudo mn --topo linear,6
\end{lstlisting}
After executing this command the network will be running and a single local controller
(called \code{c0}) will be used for all the switches.


\subsection*{Task 2: start the remote controllers}
\subsubsection*{Step 1}
Open a new terminal and move to the directory \code{/home/mininet/pox}

\subsubsection*{Step 2}
Start the first POX controller, using the component \code{forwarding.l2\_learning}
for making the OpenFlow switches act as L2 learning switches and the component
\code{openflow.of\_01} for specifying the TCP port to listen for connections on.

\lstset{basicstyle=\scriptsize\ttfamily}
\begin{lstlisting}
$ sudo ./pox.py forwarding.l2_learning openflow.of_01 --port=6634
\end{lstlisting}

\subsubsection*{Step 3}
Open a new terminal and repeat step 1 and step 2 for starting the second controller.
Remember to specify the correct TCP port, which this time will be 6635 instead of 6634.




\subsection*{Task 3: set the controller for each switch}
\subsubsection*{Step 1}
Open a new terminal and use ovs-vsctl to connect the switch \code{s3} to the POX
controller listening on port 6634 \cite{ref-8}:
\begin{lstlisting}
$ sudo ovs-vsctl set-controller s3 tcp:127.0.0.1:6634
\end{lstlisting}
On the terminal with which you started that controller you should see that a new
switch has connected, like in the screenshoot below:
\begin{figure}[htb]
	\centering
	\includegraphics[width=1\linewidth]{img/controller-connection.png}
\end{figure}


\subsubsection*{Step 2}
Use ovs-vsctl to connect the switch \code{s4} to the POX controller listening on
port 6634:
\begin{lstlisting}
$ sudo ovs-vsctl set-controller s4 tcp:127.0.0.1:6634
\end{lstlisting}
Again, on the controller's terminal you should se that the switch has connected
to the controller.


\subsubsection*{Step 3}
Apply the same proceeding showed in the first two step to connect the switches
\code{s5} and \code{s6} to the POX controller listening on port 6635.



\subsection*{Task 4: test the network}
Test the connectivity between all hosts.

\section*{Lab activity 4: final challenge}

\subsection*{Topology diagram}
\begin{figure}[htb]
	\centering
	\includegraphics[width=1\linewidth]{img/challange-topology.png}
	\caption{the topology that you will have to implement in this activity, or rather
  a tree topology with multiple SDN controllers.
  The controllers \code{c2} and \code{c3} are local while \code{c0} and \code{c1}
  are remote controllers running on the same machine on which Mininet is running,
	listening respectively on TCP ports 6635 and 6636.}
	\label{fig:challenge-topology}
\end{figure}




\subsection*{Learning objectives}
After finishing this lab activity you will be able to apply what you have learnt
through the previous lab activities in order to implement on your own a given
topology which includes multiple controllers.





\subsection*{Scenario}
In this activity you will have to implement on your own the topology shown in
figure \ref{fig:challenge-topology} using the method you think is most appropriate
among those previously showed in this paper.

The activity is meant to let you test the knowledge you acquired through the
previous three activities proposed in this paper, therefore it assumes that you
have already completed them successfully.

The solution for this activity is available in the appendix A of this paper.



\subsection*{Task 1: implement the topology}
Implement the topology shown in figure \ref{fig:challenge-topology} using the method
you think is most appropirate.

\subsection*{Task 2: fix the topology}
Assumes that the controller \code{c3} failed and it's not working anymore: without
shutting down the whole network, fix the problem by connecting the switches served by \code{c3}
to the other available remote controller.

\section*{Appendix A}
\lstset{
 upquote=true,
 showspaces=false,
 showtabs=false,
 frame=none,
 tabsize=2,
 breaklines=true,
 numbers=none,
 showstringspaces=false,
 breakatwhitespace=true,
 escapeinside={(*@}{@*)},
 keywordstyle=\bfseries,
 basicstyle=\scriptsize\ttfamily,
 moredelim=**[is][\color{red}]{@}{@},
}

In this appendix are reported the answers to the questions proposed in all the
activities included in this paper.


\subsection*{Activity 1}

\subsubsection*{Task 7 - Step 2}
\textit{Test the created topology: verify the network connectivity between all hosts.
Write in the lines below the commands you used and the results you obtained.}
\begin{figure}[htb]
	\centering
	\includegraphics[width=0.5\linewidth]{img/task-7-step-2.png}
\end{figure}



\subsubsection*{Task 7 - Step 3}
\textit{Verify that the bandwidth and the delay of each link comply with the values
specified in the topology diagram shown in figure 1. Write in the lines below
the commands you used and the results you obtained.}

\begin{itemize}
  \item \code{h3 ping h4 -c 10}

  Output:
  \begin{lstlisting}
  --- 10.0.0.2 ping statistics ---
  5 packets transmitted, 5 received, 0% packet loss, time 4006ms
  rtt min/avg/max/mdev = 23.186/25.264/26.760/1.507 ms
  \end{lstlisting}
  The average RTT is approximately 20ms, wich comply the link delays values
  specified in the topology diagram. The same proceeding can be used to test the
  delay of all the others links.

  It's important to point out that the link delay between the two switches doesn't
  comply with the topology diagram: this behavior is however acceptable and it's due to
  the fact that in Mininet the switches are by default run in the same network
  namespace inside the kernel space, therefore they communicate to each other
  without using the created link which has the performance parameters specified
  in the Python script.

  \item \code{iperf h3 h4}

  Output:
  \begin{lstlisting}
  *** Iperf: testing TCP bandwidth between h3 and h6
  *** Results: ['4.75 Mbits/sec', '6.23 Mbits/sec']
  \end{lstlisting}

  The same proceeding can be used to test the bandwidth between all nodes.
\end{itemize}



\subsubsection*{Task 8 - Question 1}
\textit{What are the advantages of having more controllers instead of
one single controller which serves all the switches of the network?}




\subsubsection*{Task 8 - Question 2}
\textit{Would the fault tolerance of the netowrk shown in figure 1
change if we used only one controller (linked to both the switches)
instead of two?}




\subsubsection*{Task 8 - Question 3}
\textit{In this activity the topology shown in figure 1 was implemented
assuming that the two controllers were local controllers. How
would you have to change the Python script you created in this
activity in order to use remote controllers instead of local ones?
(Hint: see reference [5])}

The changes that has to be applied to the Python script are marked with the red
colour in the listing below.

\lstset{
 upquote=true,
 showspaces=false,
 showtabs=false,
 frame=single,
 tabsize=2,
 breaklines=true,
 numbers=left,
 showstringspaces=false,
 breakatwhitespace=true,
 escapeinside={(*@}{@*)},
 keywordstyle=\bfseries,
 basicstyle=\scriptsize\ttfamily,
 moredelim=**[is][\color{red}]{@}{@},
}
\begin{minipage}{\linewidth}
\begin{lstlisting}
#!/usr/bin/Python
from mininet.net import Mininet
from mininet.node import Controller, OVSSwitch, @RemoteController@
from mininet.cli import CLI
from mininet.log import setLogLevel, info
from mininet.link import TCLink

def multiControllerNet():
    net = Mininet( controller=Controller, switch=OVSSwitch, link=TCLink )

    info( "*** Creating hosts\n" )
    h1 = net.addHost('h3')
    h2 = net.addHost('h4')
    h3 = net.addHost('h5')
    h4 = net.addHost('h6')

    info( "*** Creating switches\n" )
    s1 = net.addSwitch( 's1' )
    s2 = net.addSwitch( 's2' )

    info( "*** Creating links\n" )
    net.addLink( h1, s1, bw=5, delay='5ms' )
    net.addLink( h2, s1, bw=5, delay='5ms' )
    net.addLink( h3, s2, bw=5, delay='5ms' )
    net.addLink( h4, s2, bw=5, delay='5ms' )
    net.addLink( s1, s2, bw=10, delay='2ms' )

    info( "*** Creating (reference) controllers\n" )
    c0 = net.addController( 'c0', @controller=RemoteController, ip='127.0.0.1'@, port=6633 )
    c1 = net.addController( 'c1', @controller=RemoteController, ip='127.0.0.1'@, port=6634 )

    info( "*** Starting network\n" )
    net.build()
    c0.start()
    c1.start()
    s1.start( [ c0 ] )
    s2.start( [ c1 ] )

    info( "*** Running CLI\n" )
    CLI( net )

    info( "*** Stopping network\n" )
    net.stop()

if __name__ == '__main__':
    setLogLevel( 'info' )
    multiControllerNet()
\end{lstlisting}
\end{minipage}

For simplicity it has been assumed that the two remote controllers are running on the same machine on
which Mininet is running, therefore the ip \code{127.0.0.1} has been used (if the
controllers had been on different machines the relative IP should have been used).




\subsubsection*{Task 8 - Question 4}
\textit{How could we improve the fault tolerance of the netowork shown
in figure 1 making minor changes to the Python script used?}







\lstset{
 upquote=true,
 showspaces=false,
 showtabs=false,
 frame=none,
 tabsize=2,
 breaklines=true,
 numbers=none,
 showstringspaces=false,
 breakatwhitespace=true,
 escapeinside={(*@}{@*)},
 keywordstyle=\bfseries,
 basicstyle=\scriptsize\ttfamily,
 moredelim=**[is][\color{red}]{@}{@},
}
\subsection*{Activity 2}
\subsubsection*{Task 5 - Step 2}
\textit{Test the created topology: verify the network connectivity between all hosts.
Write in the lines below the commands you used and the results you obtained.}
\begin{figure}[htb]
	\centering
	\includegraphics[width=0.8\linewidth]{img/activity-2-task-5-step-2.png}
\end{figure}


\subsubsection*{Task 5 - Step 3}
\textit{Verify the bandwidth and the delay between the hosts h1s1 and h3s4. Write
in the lines below the commands you used and the results you obtained.}
\begin{itemize}
  \item \code{iperf h1s1 h3s4}

  Result:
  \begin{lstlisting}
  *** Iperf: testing TCP bandwidth between h1s1 and h3s4
  *** Results: ['4.25 Gbits/sec', '4.26 Gbits/sec']
  \end{lstlisting}

  \item \code{h1s1 ping h3s4 -c 10}

  Result:
  \begin{lstlisting}
  --- 10.0.0.12 ping statistics ---
  10 packets transmitted, 10 received, 0% packet loss, time 9006ms
  rtt min/avg/max/mdev = 0.107/4.042/24.062/7.490 ms
  \end{lstlisting}
\end{itemize}



\clearpage
\printbibliography


\end{document}
