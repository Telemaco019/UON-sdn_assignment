\documentclass[12pt, a4paper]{article}
\usepackage[utf8]{inputenc}
\usepackage{listings}
\usepackage{courier}
\usepackage{color}
\definecolor{light-gray}{gray}{0.95}

\lstset{language=[Sharp]C,
 showspaces=false,
 showtabs=false,
 frame=single,
 tabsize=2,
 breaklines=true,
 numbers=left,
 showstringspaces=false,
 breakatwhitespace=true,
 escapeinside={(*@}{@*)},
 keywordstyle=\bfseries,
 basicstyle=\scriptsize\ttfamily,
}

\newcommand{\code}[1]{{\footnotesize\texttt{#1}}}

\setlength\parindent{0pt} % Indentazione paragrafi
\setlength{\parskip}{1ex plus 0.5ex minus 0.2ex} % Spaziatura paragrafi

\title{SDN controller clustering \\ \large Computer Networks module - SDN assignment}
\author{Michele Zanotti}
\date{Spring term 2018}

\begin{document}

\begin{titlepage}
\maketitle
\end{titlepage}

\section*{Learning objectives}
After finishing this lab activity you will be able to:
\begin{itemize}
  \item Understand why clusters of controller are needed in SDN networks
  \item Understand which are the most common methods to build a cluster of controllers
  \item Implement a cluster of controllers using mininet
  \item Test the performance of a SDN network which uses a cluster of controllers
\end{itemize}


\section*{Overview}
The most popular methods for creating a cluster of controllers are:
\begin{itemize}
  \item by using the ovs-vsctl tool
  \item by python script
\end{itemize}
Cluster of controlers can be:
\begin{itemize}
  \item physically centralized
  \item physically distributed
\end{itemize}


\section{Task 1: build a cluster of local controllers}
In this task we will create a simple network topology with multiple local controllers
that make up a cluster (which act as a single entity?).

The topology is the simple linear topology shown in figure *** and it will be
created with a python script using the middle-level API provided by mininet. The following
script will be used as a template for writing the complete script:

\begin{minipage}{\linewidth}
\begin{lstlisting}
#!/usr/bin/python
from mininet.net import Mininet
from mininet.node import Controller, OVSSwitch
from mininet.cli import CLI
from mininet.log import setLogLevel, info

def multiControllerNet():
    net = Mininet( controller=Controller, switch=OVSSwitch )

    "" Create hosts ""



    "" Create switches ""



    "" Create controllers ""



    info( "*** Running CLI\n" )
    CLI( net )

    info( "*** Stopping network\n" )
    net.stop()

if __name__ == '__main__':
    setLogLevel( 'info' )  # for CLI output
    multiControllerNet()
\end{lstlisting}
\end{minipage}

\subsection*{Step 1: create a new python script}
Create a new python script called \emph{controllers-1.py} and paste into it the
template script shown above. The script simply create a new minine network
and starts the mininet CLI: we will use it as a start point and in the next steps
and we will add code for creating hosts, switches, links and controllers.



\subsection*{Step 2: add host to the network}
In the host section of the script add the following lines:

 \code{h1 = net.addHost('h3')} \\
 \code{h1 = net.addHost('h4')} \\
 \code{h1 = net.addHost('h5')} \\
 \code{h1 = net.addHost('h6')}

The first line prints to the console that hosts are being created while the others
lines add four new hosts to the network.



\subsection*{Step 3: add switches to the network}
Now we need to add the two switches to the network, therefore we have to add the
following code to the respective section of the template script:

\code{info( "*** Creating switches\textbackslash n" )} \\
\code{s1 = net.addSwitch( 's1' )} \\
\code{s2 = net.addSwitch( 's2' )}

As in the step 2, the first line print to the console a message to inform the user
that the switches are being created, while the next two lines add two new switches
to the network.



\subsection*{Step 4: create links between nodes}
We have to add the links between hosts and switches and the link between the two
switches. For doing that we can add the following lines:

\code{net.addLink( h3, s1 )} \\
\code{net.addLink( h4, s1 )} \\
\code{net.addLink( h5, s2 )} \\
\code{net.addLink( h6, s2 )} \\
\code{net.addLink( s1, s2)}

\subsection*{Step 5: create controllers}




\end{document}
