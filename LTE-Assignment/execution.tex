\section{Execution phase}



\subsection*{Step 1}
The source MME completes the preparation phase by sending to the source eNodeB
the \emph{Handover command} message through the interface S1. The message
contains the following parameters:
\begin{itemize}
    \item Target to Source Transparent Container
    \item E-RABs to Release List
    \item Bearers Subject to Data Forwarding List: if ``Direct
    Forwarding'' applies it is the list of ``Address(es) and TEID(s) for user
    traffic data forwarding'' received from the target SGSN during step 7
    of the preparation phase, otherwise if ``Indirect Forwarding'' applies
    it contains the parameters received in step 8a from the S-GW
    \item E-RABs to Release List
\end{itemize}




\subsection*{Step 2}
The source eNodeB initiates data forwarding for bearers specified in the
``Bearers Subject to Data Forwarding List'' of the message received from the MME.

After that, the eNodeB sends the E-UTRAN command \emph{HO} to the UE for telling
it to handover to the target access network. This message includes a transparent
container which contains the radio parameters that the RNC has set-up during the
preparation phase.

After the reception of the HO command the UE has to associate its bearer IDs to
the respective RABs according to the relation with the NSAPIs\footnote{NSAPI =
Network Service Access Point Identifier, it is used to identify PDP contexts in
the SGSN. A PDP context is a data structure which contains subscriber's session
information such as its IMSI and its IP address} and has to suspend
the uplink transmission of user data.



\subsection*{Step 3}
Void




\subsection*{Step 4}
The UE executes the handover to the target UTRAN according to the prameters
contained in the message received in step 2. At this point it can resume the user
data transfer only for those NSAPIs which have been associated to a RAB, namely
the NSAPIs for which there are radio resources allocated in the target RNC.




\subsection*{Step 5}
After the RNC-ID and the S-RNTI\footnote{S-RNTI = Serving RNC Radio Network
Temporary Identifier, in UMTS the S-RNTI is the UE identifier which is allocated
by the RNC and it's unique within that RNC} are exchanged with the UE, the target
RNC sends the \emph{Relocation Complete} message to the target SGSN, indicating
therefore the completion of the relocation from the source E-UTRAN to the target
RNC. After receiving this message, the SGSN is ready for receiving data from
the target RNC. 
