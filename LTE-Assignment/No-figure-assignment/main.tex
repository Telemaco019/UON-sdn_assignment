\documentclass[12pt, a4paper]{article}
\usepackage[utf8]{inputenc}
\usepackage{listings}
\usepackage{courier}
\usepackage{fancyhdr}
\usepackage{color}
\usepackage{xcolor}
\usepackage{caption}
\usepackage{graphicx}
\usepackage{url}
\usepackage{textcomp}
\usepackage{pdfpages}
\usepackage[colorlinks]{hyperref}
\hypersetup{urlcolor=blue}




\setlength\parindent{0pt} % Indentazione paragrafi
\setlength{\parskip}{1ex plus 0.5ex minus 0.2ex} % Spaziatura paragrafi

\begin{document}







\clearpage
\section*{List of Abbreviations}
\begin{tabular}{ p{4cm} l }
 3GPP & Third Generation Partnership Project \\
 CN & Core network \\
 CSG & Closed subscriber group (of a home eNodeB) \\
 E-UTRAN & Evolved UMTS terrestrial radio access network \\
 EPS & Evolved packet system \\
 GGSN & Gateway GPRS Support Node \\
 GTP & GPRS tunnelling protocol \\
 GTP-C & GPRS tunnelling protocol control part \\
 GTP-U & GPRS tunnelling protocol user part \\
 IMEI & International mobile equipment identity \\
 LTE & Long term evolution \\
 MME & Mobility management entity \\
 NSAPI & Network Service Access Point Identifier \\
 PDN & Packet data network \\
 PDU & Protocol data unit \\
 P-GW & Packet data network gateway \\
 RAB & Radio access bearer \\
 RNC & Radio network controller \\
 SDU & Service data unit \\
 SGSN & Serving GPRS Support Node \\
 S-GW & Serving gateway \\
 S-RNTI & Serving RNC Radio Network Temporary Identifier \\
 TEID & Tunnel endpoint identifier \\
 UE & User equipment \\
 UMTS & Universal Mobile Telecommunications System \\
 UTRAN & UMTS terrestrial radio access network \\


\end{tabular}

\clearpage
\section{Introduction}
The intra-RAT handover from E-UTRAN to UTRAN is composed of two phases, each of
which involves (as well as the UE) either nodes from the LTE network and nodes
from the UMTS netowork: the \emph{preparation phase} and the \emph{execution phase}.
In particular, the LTE elements involved in the procedure are:
\begin{itemize}
  \item eNodeB
  \item E-UTRAN network
  \item MME
  \item S-GW
  \item P-GW
\end{itemize}
while the UMTS elements are:
\begin{itemize}
  \item RNC
  \item SGSN
  \item GGSN
\end{itemize}

It is assumed that both the LTE and the UMTS networks belong to the same
operator. This means that the P-GW of the LTE network and the GGSN of the UMTS
network are the same physical device, therefore, since in this paper the focus
is on the LTE network, only the P-GW will be showed. Moreover, since the goal
of this paper is to show only the intra-RAT handover procedure, some non-central
aspects such as the relocation of the S-GW will not be considered and the related
messages and parameters will be left out.

It is important to point out that before, during and after the handover the UE
is in the state ``LTE\_ACTIVE'' and user data is uploaded/downloaded (otherwise
if the UE was in idle state it wouldn't be a handover but a simple cell reselection).
In particular, as shown in figure \ref{fig:pre-handover-links}, before the handover procedure both uplink
and downlink user data is transmitted through:
\begin{itemize}
	\item bearer(s) between UE and source eNodeB
	\item GTP\footnote{GPRS Tunneling protocol (GTP) is a IP/UDP based protocol used to
	encapsulate user data when passing through core network (GTP-U) or to carry bearer
	specific signalling traffic between various core network nodes (GTP-C).}
  tunnel(s) between	source eNodeB, S-GW and P-GW.
\end{itemize}


The figure \ref{fig:architecture-model} shows the elements involved in this
handover procedure, how they are connected to each other and through which
interfaces, figure \ref{fig:pre-handover-links} shows how data is transmitted
before the start of the procedure.



\section{Preparation phase}
\subsection*{Step 1}
The source eNodeB decides to initiate an Inter-RAT handover to the target access network.



\subsection*{Step 2}
The source eNodeB sends a \emph{Handover Required} message to the
source MME, requesting the CN to establish resources in the target RNC and in the
target SGSN. The message is sent through the S1 interface and it contains the
following parameters:
\begin{itemize}
	\item S1AP Cause: it specifies the reason of the message
	\item Target RNC Identifier: it identifies the target RNC
	\item CSG access mode: included only if the target cell is a hybrid cell
	\item CSG ID: included only if the target cell is a CSG or hybrid cell, it identifies the cell
	\item Source to Target Transparent Container: it carries RRC parameters and
	Radio Bearer information necessary to set-up the radio bearers.
\end{itemize}


\subsection*{Step 3}
The source MME determines from the ``Target RNC Identifier'' field that
the type of handover is intra-RAT Handover to UTRAN Iu mode and, if the CSG ID
is included in the message, it checks the UE's CSG subscribtion. If the UE isn't
subscribed to the CSG, then the MME rejects the handover, unless the UE has
emergency bearer services ongoing (in this case the handover to the target RNC is
performed independent of the restrictions).	If the handover isn't rejected then
the MME selects the target SGSN and sends to it	a \emph{Forward Relocation Request}
message through the S3 interface. Some of the parameters included in this message are:
\begin{itemize}
	\item user IMSI
	\item ISR Supported: it indicates if the source MME and the source S-GW are
	able to activate ISR\footnote{ISR = ``idle mode signalling reduction''. When
	this mode is active the network can simultaneously register the UE in a
	routing area that is served by an SGSN and in one or more tracking areas
	that are served by an MME.}
	\item PDN connections: it indicates the active PDN connections
	\item RAN cause: it's the S1AP cause received from the eNodeB
	\item CSG ID: included only if the target cell is a CSG cell or a hybrid cell
	\item CSG Membership Indication: it indicates if the UE is a CSG member. It's
	included only if the target cell is a hybrid cell or if it is a CSG cell and
	there is at least one emergency bearer service
	\item MM context: it includes information on the EPS Bearer\footnote{EPS Bearers
	= Bearers between the UE and the P-GW} contexts
	\item Source to Target Transparent Container
\end{itemize}
 If none of the UE's EPS Bearers is supported by the target SGSN, the source MME
 rejects the handover attempt and sends a Handover Preparation Failure message to the
 Source eNodeB (see chapter Handover Reject).


\subsection*{Step 4}
The target SGSN establishes the EPS Bearer contexts indicated by the message
received from the MME and deactivates the Bearer contexts which can't be
estabilished.

After that, it requests the target RNC to establish the radio
network resources (RABs) by sending the
\emph{Relocation Request} message through the Iu-PS interface. Some of the
parameters included in this message are:
\begin{itemize}
	\item Encryption information: it is sent in order to allow data transfer to
	continue in the new UTRAN target cell without requiring a new Authentication
	and Key Agreement (AKA) procedure
	\item RAB to be setup list: for each RAB to be set up it contains information
	such as the RAB ID (which contains the NSAPI value) and other RAB parameters
	\item CSG ID and CSG Membership Indication: included only when provided by
	the the source MME in the \emph{Forward Relocation Request} message
	\item Source RNC to Target RNC Transparent Container: it includes the information
	received from the source eNodeB included in the Source to Target Transparent
	Container field of the \emph{Handover required} message
\end{itemize}
If the target cell is a CSG cell, the target RNC verifies the CSG ID provided
by the target SGSN and rejects the handover if it does not match the CSG ID for
the target cell. If the CSG Membership Indication is ``non member'', the target
RNC only accepts emergency bearers.



\subsection*{Step 4a}
For each accepted bearer, the target RNC allocates radio and Iu
user plane resources. After that, the target RNC sends back to the SGSN the
\emph{Relocation Request Acknowledge} message, which contains a list of the
setup bearers and a list of the failed to setup bearers, which will be deactivated
by the SGSN.

After sending the ACK the RNC is prepared to receive downlink GTP
PDUs\footnote{Packets received by a protocol layer are called SDU while packets
output of a layer are called PDU.}
from the S-GW (or from the target SGSN if Direct Tunnel is not used) for the
accepted bearers.


%\subsection*{Step 6}
%\begin{figure}[!htb]
%	\centering
%	\includegraphics[width=0.9\linewidth]{img/6.png}
%	\label{fig:6}
%\end{figure}
%f Indirect Forwarding and relocation of S-GW apply then the
%target SGSN sends a \emph{Create Indirect Data Forwarding Tunnel Request}
%message to the target S-GW through the S4 interface. If GTP Direct Tunnel is used
%then message includes the Target RNC Address, while if Direct Tunnel isn't
%used the message includes the SGSN address. In both cases the message also
%include the TEIDs\footnote{TEID = Tunnel Endpoint identifier. Each GTP tunnel
%is associated with two TEID: one for the downlink and one for the uplink}
%for downlink user data forwarding.



%\subsection*{Step 6a}
%The S-GW responds to the SGSN with a \emph{Create Indirect Data Forwarding Tunnel
%Response} message, specifying as parameter the S-GW Address(es) and
%the S-GW DL TEID(s) for data forwarding.



\subsection*{Step 5}
The target SGSN sends the \emph{Forward Relocation Response} message
to the source MME through the S3 interface. Some of the parameters contained in
the message are:
\begin{itemize}
 \item Target to Source Transparent Container: it contains the value of the
 Target RNC to Source RNC Transparent Container received from the target RNC
 \item Address(es) and TEID(s) for User Traffic Data Forwarding: this field
 define the destination tunnelling endpoint for user data forwarding. If Direct Tunnel
 is used then it contains the addresses and GTP-U tunnel endpoint parameters
 to the Target RNC, otherwise it contains the DL GTP-U tunnel endpoint parameters
 to the Target SGSN.
 \item SGSN Tunnel Endpoint Identifier for Control Plane
 \item SGSN Address for Control Plane
\end{itemize}



\subsection*{Step 6}
If ``Indirect Forwarding'' applies, the Source MME sends the message \emph{Create
Indirect Data Forwarding Tunnel Request} to the S-GW used for indirect forwarding.
The parameters contained in the message are the list of ``Address(es) and TEID(s) for Data
Forwarding'' received in step 5 and the EPS Bearer ID(s).



\subsection*{Step 6a}
The S-GW replies sending message \emph{Create Indirect Data Forwarding Tunnel
Response}, which contains the S-GW Address(es) and the TEID(s) for data
forwarding. Note that the  Indirect Forwarding may be performed via a S-GW
which is different from the S-GW used as the anchor point for the UE.
If the S-GW doesn't support data forwarding, the message contains only an appropriate
cause.

\section{Execution phase}



\subsection*{Step 1}
The source MME completes the preparation phase by sending to the source eNodeB
the \emph{Handover command} message through the interface S1. The message
contains the following parameters:
\begin{itemize}
    \item Target to Source Transparent Container
    \item E-RABs to Release List
    \item Bearers Subject to Data Forwarding List: if ``Direct
    Forwarding'' applies it is the list of ``Address(es) and TEID(s) for user
    traffic data forwarding'' received from the target SGSN during step 5
    of the preparation phase (addresses and GTP-U tunnel endpoint parameters to
    the Target RNC or to the target SGSN if direct tunnel is used), otherwise
    if ``Indirect Forwarding'' applies it contains the parameters received in
    step 6a (S-GW addresses)
\end{itemize}



\subsection*{Step 2}
The source eNodeB initiates data forwarding for bearers specified in the
``Bearers Subject to Data Forwarding List'' of the message received from the MME.

After that, the eNodeB sends the E-UTRAN command \emph{HO} to the UE for telling
it to handover to the target access network. This message includes a transparent
container which contains the radio parameters that the RNC has set-up during the
preparation phase. The message is sent through the DCCH logical channel, which
maps to the DL-SCH transport channel, which maps to PDSCH physical channel \cite{channels}.

After the reception of the HO command the UE has to associate its bearer IDs to
the respective RABs according to the relation with the NSAPIs\footnote{NSAPI is
used to identify PDP contexts in the SGSN. A PDP context is a data structure
which contains subscriber's session information such as its IMSI and its IP
address} and has to suspend the uplink transmission of user data.



\subsection*{Step 3}
The UE executes the handover to the target UTRAN according to the prameters
contained in the message received in step 2. At this point it can resume the user
data transfer only for those NSAPIs which have been associated to a RAB, namely
the NSAPIs for which there are radio resources allocated in the target RNC.



\subsection*{Step 4}
After the RNC-ID and the S-RNTI\footnote{In UMTS the S-RNTI is the UE identifier which is allocated
by the RNC and it's unique within that RNC} are exchanged with the UE, the target
RNC sends the \emph{Relocation Complete} message to the target SGSN, indicating
therefore the completion of the relocation from the source E-UTRAN to the target
RNC. After receiving this message, the SGSN is ready for receiving data from
the target RNC.



\subsection*{Step 5}
The target SGSN informs the source MME that the UE has arrived to the target side
(UMTS network) by sending the Forward Relocation Complete Notification message.

When the MME receives the message it starts a timer and it releases all the bearers
that were not included in the Forward Relocation Request message sent in step 3
of the preparation phase by sending a Delete Bearer Command to the S-GW.

When the timer expires, if ISR Activated was not indicated in the message received
from the SGSN and Indirect Forwarding is used (and therefore the SGSN allocated
S-GW resources for it), the MME releases all the bearer resources of the UE by
sending Delete Indirect Data Forwarding Tunnel Request message to the S-GW used
for indirect forwarding (see step 9).



\subsection*{Step 6}
The SGSN informs the S-GW that now the SGSN is responsible for all the EPS Bearer
contexts that the UE has established by sending to it a Modify Bearer request per
each PDN connection. The message contains the following parameters:
\begin{itemize}
	\item SGSN address and Tunnel Endpoint Identifier for Control Plane
	\item NSAPI(s)
	\item SGSN Address(es) and TEID(s) for User Traffic for the accepted EPS bearers
	(if Direct Tunnel is not used)
	\item RNC Address(es) and TEID(s) for User Traffic for the accepted EPS bearers
	(if Direct Tunnel is used)
	\item RAT type
	\item ISR Activated
	\item User location information (only if the SGSN supports location information
	change reporting)
\end{itemize}
??? The SGSN releases the non-accepted EPS Bearer contexts by triggering the Bearer Context deactivation
procedure. If the Serving GW receives a DL packet for a non-accepted bearer, the Serving GW drops the DL
packet and does not send a Downlink Data Notification to the SGSN. ???




\subsection*{Step 7}
The S-GW sends to the SGSN the Modify Bearer Response message for acknowledging
the user plane switch to the target SGSN.

After this step the user data path is finally established and it's shown in
figure \ref{fig:final-data-path}.




\subsection*{Step 8}
When the UE recognises that its current Routing Area is not registered with the
network, or when the UE's TIN indicates "GUTI", the UE initiates a Routing Area
Update procedure with the SGSN. This procedure is the UMTS equivalent of the
Tracking Area Update procedure in LTE and it's described in reference
\cite{routing-area-update}.




\subsection*{Step 9}
If indirect forwarding was used then when the timer started by the MME at step 5
expires the MME sends a Delete Indirect Data Forwarding Tunnel Request message
to the S-GW for releasing the temporary resources used for indirect forwarding.

\section{Handover reject}
\begin{figure}[htb]
	\centering
	\includegraphics[width=1\linewidth]{img/handover-reject.png}
	\label{fig:preparation-phase}
\end{figure}

The Target RNC may reject the Handover if none of the RABs specified in the
Relocation Request message could be established. In this case no UE context is
established in the SGSN/RNC and no resources are allocated, the UE therefore
remains in the Source eNodeB/MME.



\subsection*{Step 1}
Step 1 to 4 are identical to the ones shown in the first chapter (Execution phase).



\subsection*{Step 4a}
The RNC fails to allocate any resources for any of the requested RABs, therefore
it sends to the SGSN a Relocation Failure message. When the SGSN receives the
Relocation Failure message it clears any reserved resources for this UE.



\subsection*{Step 5}
The SGSN sends the Forward Relocation Response message to the Source MME, specifying
the handover reject as parameter in the message (cause field).



\subsection*{Step 6}
When the MME receives the Forward Relocation Response message it sends a Handover
Preparation Failure message to the Source eNodeB.



\begin{thebibliography}{99}
\footnotesize

  \bibitem{direct-tunnelling}
  Cisco. (2017).
  \textit{P-GW Administration Guide, StarOS Release 20 - Direct Tunnel for 4G (LTE) Networks [Cisco ASR 5000 Series]}. [online]
  Available at: \url{https://www.cisco.com/c/en/us/td/docs/wireless/asr_5000/20/P-GW/b_20_PGW_Admin/b_20_PGW_Admin_chapter_011111.html}
  [Accessed 5 May 2018].

  \bibitem{routing-area-update}
  3GPP TS 23.401 (September 2011)
  \textit{General Packet Radio Service (GPRS) Enhancements for Evolved Universal
  Terrestrial Radio Access Network (E-UTRAN) Access},
  Release 10,
  sections 5.3.3.3, 5.3.3.6.

  \bibitem{channels}
  Poole, I. (n.d.).
  \textit{LTE Physical, Logical and Transport Channels :: Radio-Electronics.Com}. [online]
  Radio-electronics.com.
  Available at: \url{http://www.radio-electronics.com/info/cellulartelecomms/lte-long-term-evolution/physical-logical-transport-channels.php}
  [Accessed 16 May 2018].


\end{thebibliography}


\end{document}
